\documentclass{physor2012}
%
%  various packages that you may wish to activate for usage 
%\usepackage{tabls}
\usepackage{cites}
\usepackage{epsf}
%\usepackage{appendix}
%
% Define title...
%
\title{GENERATING NUCLIDE-SPECIFIC SCATTERING KERNELS WITH SERPENT}
%
% ...and authors
%
\author{%
  % FIRST AUTHORS 
  %
  \textbf{Anthony Scopatz and Erich Schneider} \\
  The University of Texas at Austin \\
  1 University Station, C2200, Austin, TX, 78712 \\
  \url{scopatz@mail.utexas.edu}; \url{eschneider@mail.utexas.edu} \\
}
%
% Insert authors' names and short version of title in lines below
%
\newcommand{\authorHead}      % Author's names here
           {A. M. Scopatz and E. A. Schneider}  
\newcommand{\shortTitle}      % Short title here
           {Gneerating Nuclide-Specific Scattering Kernels with Serpent}


%%%%%%%%%%%%%%%%%%%%%%%%%%%%%%%%%%%%%%%%%%%%%%%%%%%%%%%%%%%%%%%%%%%%%
%
%   BEGIN DOCUMENT
%
%%%%%%%%%%%%%%%%%%%%%%%%%%%%%%%%%%%%%%%%%%%%%%%%%%%%%%%%%%%%%%%%%%%%%
\begin{document}
\maketitle
\begin{abstract}
  Use 8.5$\times$11 paper size, with 1'' margins on all sides.  A required 200-250 
  word abstract starts on this line.  Leave two blank lines before ``ABSTRACT''
  and one after.  Use 10 point Times New Roman here and single 
  spacing. The abstract is a very brief summary highlighting main 
  accomplishments, what is new, and how it relates to the state-of-the-art.
\end{abstract}
\keywords{List of at most five key words}


\section{INTRODUCTION} 
Creating full-core, homogenized neutron group-to-group scattering cross sections
for nuclear power reactors is traditionally done via Monte Carlo methods (such as
MCNPX \cite{}).  Computing region or sub-region kernels may be performed in much 
the same way.  However, these methods homogenize not only over reactor geometry
but over material composisition as well.  For perturbation-based reactor models 
\cite{}, nuclide specific scattering cross sections are needed.

Perturbation-based reactor models (such as those in Bright \cite{}) use a 
set of precomputed libraries (such as those generated by Char \cite{}).
These libraries are comprised of microscopic neutron cross sections 
or reaction rates for each nuclide for a given reactor state.  For reactor 
specifications that are not one of the precomputed states, two or more libraries
are interpolated at run-time to form a dynamically generated core.  This mechanism
grants large performance gains over transport methods while maintatining 
acceptable core performance fidelity.

For multi-energy group reactor models using the above algorithm, the scattering 
cross section $\sigma_{s,g\to h,i}$ [barns] is needed to be able to perform 
consistent interpolations. (Here, the subscript $s$ denotes the scattering channel, 
the subscripts $g$ \& $h$ index $G$ incident and exiting energies, and the subscript
$i$ indexes $I$ species.)  This parameter is notbaly more important in thermal reactors
than fast cores.  Furthermore, previous single energy group studies \cite{} did 
not require the group-to-group scattering cross section at all because both 
$g$ \& $h$ are intergated over.

While most Monte Carlo neutron transport codes may compute few group cross
sections (\emph{i.e.} $\sigma_{s,g,i}$), most fail to expose the distribution
of neutron exiting energies after a reaction ($\sigma_{s,g\to h,i}$).  Those
codes which do report exiting energies display only the sum over all consitiuent 
materials in the region of interest ($\sigma_{s,g\to h}$).  Thus obtaining 
the parameters for a perturbation-based reactor model is itself a challenge.

In the initial scoping of library generation (Char) two burnup-coupled transport 
codes were considered: MCNPX (v2.6+) and Serpent (v1.7+).  MCNPX confers the 
ability to compute $\sigma_{s,g,i}$.  Serpent by default will calculate 
$\sigma_{s,g\to h}$ and allows the user to compute $\sigma_{s,g,i}$ as well.

Why serpent and what I did.

\section{SECOND OR SUBSEQUENT MAJOR HEADING} 
\label{sec:first}
%
This is Section~\ref{sec:first}. It is followed by a subsection, that is, 
\ref{sec:second}. The style for subsection titles and all text in this template is defined in 
the \emph{physor2012.cls} file.  Make sure to avoid widow/orphan lines.
%
\subsection{Subsection Title: First Character of Each Non-trivial Word is Uppercase} 
\label{sec:second}
%
Double-space before and after secondary titles is automatic.  Figures and 
tables should appear as close as possible to where they are first
cited, e.g., Fig.~\ref{fig:amdahl}, in the text.  Figures are numbered in Arabic 
numerals, with the caption centered below the figure, in \textbf{boldface}.  
Triple-space before the figure, and after the figure caption.

\begin{figure}[!htb]
  \centering
  %\includegraphics[scale=0.60]{./figure.pdf}
  \caption{\bf SCALE/TRITON-NEWT Model of BWR Assembly in Order to Show and Example of a Figure and a Multi-line Caption} 
  \label{fig:amdahl}
\end{figure}

When importing figures or any graphical image please verify two things:
\begin{itemize}
\item Any number, text or symbol is in Times font and is not smaller than 
  10-point after reduction to the actual window in your paper
\item That it can be translated into PDF
\end{itemize}
Equations, such as Eq. (\ref{sample_equation}), should be centered and 
sequentially numbered to the flush right of the formula.

\begin{equation}
  \label{sample_equation}
  \mathrm{Speedup}=\frac{1}{\frac{f}{p}+(1-f)}
\end{equation}

The continuation of a paragraph after an equation should not be indented.  
All paragraphs, as well as section or subsection headings, are separated by 
just one single empty line.
%
\subsubsection{Sub-subsection level and lower: only first character uppercase}
%
See Table \ref{table:example} for a sample table.  The ``tabls'' package is
recommended for improved row and column spacing.  Notice the caption appears 
above the table by setting the \verb!\caption! command immediately 
after the \verb!\begin{table}!. Tables are numbered in Roman 
numerals, with the caption centered above the table, in \textbf{boldface}.  
Triple-space before and after the table.
%
\begin{table}[!htb]
  \centering
  \caption{\bf Parallel Performance for the Sample Problem}
  \label{table:example} 
  \begin{tabular}{||r||c|c|c||} \hline \hline
    \multicolumn{1}{||c||}{Number of} &
    \multicolumn{1}{c|}{Wall-Clock} &
    \multicolumn{1}{c|}{Speedup} &
    \multicolumn{1}{c||}{Efficiency} \\
    \multicolumn{1}{||c||}{Processors} &
    \multicolumn{1}{c|}{Time$^{a}$ (min)} &
    \multicolumn{1}{c|}{(T$_{s}$/T$_{p}$)} &
    \multicolumn{1}{c||}{(\%)} \\ \hline\hline
    \ 1 &  100.0 & \ ---    & ---  \\ \hline
    \ 2 &   52.6 & \ 1.9    & 95.0 \\ \hline \hline
  \end{tabular}
\end{table}
%
\section{CONCLUSIONS}
%
Present your summary and conclusions here.
%
\section*{ACKNOWLEDGEMENTS}
%
Acknowledge the help of colleagues and sources of funding, as appropriate.
%
The format for this template was adapted from the template for Math and Computation 2009 
posted on the Internet.  Most of the \LaTeX\ format definitions contained
in this and the accompanying files were adapted from sample files supplied 
by J. Wagner (ORNL) and James Warsa (LANL). Their help is greatly appreciated.

%
\setlength{\baselineskip}{12pt}
\bibliographystyle{physor}
\bibliography{library}

\end{document}
