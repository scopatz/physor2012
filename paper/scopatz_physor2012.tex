\documentclass{physor2012}
%
%  various packages that you may wish to activate for usage
%\usepackage{tabls}
\usepackage{cites}
\usepackage{epsf}
%\usepackage{appendix}
%
% Define title...
%
\title{INTERPOLATIONS OF NUCLIDE-SPECIFIC SCATTERING KERNELS \\
       GENERATED WITH SERPENT}
%
% ...and authors
%
\author{%
  % FIRST AUTHORS
  %
  \textbf{Anthony Scopatz and Erich Schneider} \\
  The University of Texas at Austin \\
  1 University Station, C2200, Austin, TX, 78712 \\
  \url{scopatz@mail.utexas.edu}; \url{eschneider@mail.utexas.edu} \\
}
%
% Insert authors' names and short version of title in lines below
%
\newcommand{\authorHead}      % Author's names here
           {A. M. Scopatz and E. A. Schneider}
\newcommand{\shortTitle}      % Short title here
           {Interpolations of Nuclide-Specific Scattering Kernels Generated with Serpent}


%General Short-Cut Commands
\newcommand{\superscript}[1]{\ensuremath{^{\textrm{#1}}}}
\newcommand{\subscript}[1]{\ensuremath{_{\textrm{#1}}}}
\newcommand{\nuc}[2]{\superscript{#2}{#1}}



%%%%%%%%%%%%%%%%%%%%%%%%%%%%%%%%%%%%%%%%%%%%%%%%%%%%%%%%%%%%%%%%%%%%%
%
%   BEGIN DOCUMENT
%
%%%%%%%%%%%%%%%%%%%%%%%%%%%%%%%%%%%%%%%%%%%%%%%%%%%%%%%%%%%%%%%%%%%%%
\begin{document}
\maketitle
\begin{abstract}
    The neutron group-to-group scattering cross section is an essential
    input parameter for any multi-energy group physics model.  However,
    if the analyst prefers to use Monte Carlo transport to generate group 
    constants this data is difficult to obtain for a single species of a material.  
    Here, the Monte Carlo code Serpent was modified to return the group transfer 
    probabilities on a per-nuclide basis.  This ability is demonstrated in conjunction with
    an essential physics reactor model where cross section perturbations
    are used to dynamically generate reactor state dependent group constants via 
    interpolation from pre-computed libraries.  The modified version of Serpent was 
    therefore verified with three interpolation cases designed to test the resilience 
    of the interpolation scheme to changes in intragroup fluxes.  For most species, 
    interpolation resulted in errors of less than 5\% of transport-computed values.  
    For important scatterers, such as \nuc{O}{16}, errors less than 2\% were observed.
    For nuclides with high errors ($>10\%$), the scattering channel typically
    only had a small probability of occurring.
\end{abstract}
\keywords{essential physics, neutron scattering, serpent}


\section{INTRODUCTION}
Deterministic or Monte Carlo transport methods are used to prepare neutron 
group-to-group scattering cross sections for subsequent full-core analyses.  
Region or sub-region scattering kernels are computed in much the same way.  
However, in the Monte Carlo transport applications that are the focus of 
this paper, these methods homogenize not only over reactor geometry but 
over material composition as well.  Multigroup solvers that employ a 
perturbation-based approach to generate group constants from a small 
number of pre-computed cases, such as \cite{ScopatzAnthony2011}, require 
nuclide specific scattering cross sections.

The perturbation-based multigroup solver used in this work, Bright \cite{Scopatz2011c}, uses a
set of pre-computed libraries generated by a utility called Char \cite{ScopatzAnthony2011}.
These libraries are comprised of self-shielded, intragroup flux-weighted microscopic 
neutron cross sections for each nuclide for a suite of reactor states.  For reactor
specifications that are not one of the pre-computed states, two or more libraries
are interpolated at run-time to form a dynamically generated core.  This mechanism
grants large performance gains over direct use of the `nearest' pre-computed library.

For multi-energy group reactor models using the above algorithm, the group-to-group scattering
cross sections $\sigma_{s,g\to h,i}$ [barns] are among the required group constants. 
Here, the subscript $s$ denotes the scattering channel,
the subscripts $g$ \& $h$ index $G$ incident and exiting energies, and the subscript
$i$ indexes $I$ species.  The scattering kernel is notably more important in thermal reactors
than those with a fast spectrum.  Note that single energy group Bateman equation solvers 
\cite{Croff2002} do not require the group-to-group scattering cross section because both
$g$ \& $h$ are integrated over.

While most Monte Carlo neutron transport codes may compute few group cross
sections (\emph{i.e.} $\sigma_{s,g,i}$) they do not expose the distribution
of neutron exiting energies after a scattering reaction ($\sigma_{s,g\to h,i}$).
Those codes which do report exiting energies display only the sum over all constituent
materials in the region of interest ($\sigma_{s,g\to h}$).  Thus obtaining
the input parameters for a perturbation-based reactor model is itself a challenge.

In the initial scoping of the library generation code Char two burnup-coupled transport
codes were considered: MCNPX (v2.6+) \cite{Pelowitz2008} and Serpent (v1.7+) \cite{Lepp2011}.  
MCNPX confers the ability to compute $\sigma_{s,g,i}$.  By default, Serpent will calculate
$\sigma_{s,g\to h}$ for a region and $\sigma_{s,g,i}$ for each isotope, but it does not 
produce $\sigma_{s,g\to h,i}$ for each isotope.

Because the group-to-group scattering cross section was already natively supported,
Serpent was ultimately chosen to satisfy the group constant generation needs of
Char.  However, the source code of Serpent needed to be altered to support
the reporting of $\sigma_{s,g\to h,i}$.  This modified version is available upon
request from the author\footnote{This is subject to the same license and restrictions
as Serpent itself.}.

This study presents a verification of the interpolation scheme used in the multigroup 
model which focuses on the group-to-group scattering cross section as calculated via
the modified Serpent.  Here the fuel pin radius is chosen as the physical parameter on 
which the dependence of the group constants is to be quantified.  It is set to two 
different values and $\sigma_{s,g\to h,i}$ is calculated for all nuclides using the 
modified Serpent code.  Values for the scattering kernel for intermediate fuel radii 
were then found via interpolation.  These two data sets were then compared.

In \S \ref{sec:methodology} is an explanation of the modifications made to Serpent
and the approach used in the comparison study.  In \S \ref{sec:results} the
results of this study are presented.  Lastly, \S \ref{sec:conclusions}
discusses concluding remarks and potential future work.

\section{METHODOLOGY}
\label{sec:methodology}

This work is composed of two main parts.  The first portion describes how
Serpent was altered to generate nuclide specific scattering kernels.
The second portion outlines a fuel radius perturbation study used to
validate the changes made to the Serpent code base and interpolations of the
generated scattering cross sections.

\subsection{Serpent Modifications}

The reactor physics code Serpent calculates group transfer probabilities and
group-to-group scattering cross sections for user-specified geometries and energy
group structures as part of its standard suite of results.   Call $P_{g\to h}$
the group transfer probability.  To within computational error, the following
relations are true:

\begin{equation}\label{P_h_norm}\sum_h^G P_{g\to h} = \mathbf{\vec{1}}\end{equation}
\begin{equation}\label{P_sig_norm}P_{g\to h} = \frac{\sigma_{s,g\to h}}{\sigma_{s,g}}\end{equation}

Serpent, like other physics codes, allows users to specify materials.
Denote $N_i$ as the number density [$i$-atoms/cm\superscript{3}] of the $i$\superscript{th} 
nuclide in a material and $N$ as the number density [atoms/cm\superscript{3}] of the material 
itself.  They are defined as in equation \ref{num_dens} with
$\rho_i$ [kg$_i$/cm\superscript{3}] as the $i$\superscript{th} mass density, $N_A$ 
being Avogadro's number, and MW$_i$ as the $i$\superscript{th} molecular weight.
\begin{equation}
\label{num_dens}
N_i = \frac{\rho_i N_A}{\mbox{MW}_i} 
\end{equation}
The scattering cross section of the material is thus the weighted sum of the 
constituent species:
\begin{equation}
\label{mat_norm}
\sigma_{s,g\to h} = \sum_i^I \frac{N_i}{N} \cdot \sigma_{s,g\to h,i}
\end{equation}

By combining equations \ref{P_sig_norm} \& \ref{mat_norm}, an expression for the
group transfer probability is obtained which represents the method by which it
is calculated within Serpent.
\begin{equation}
\label{P_serp}
P_{g\to h} = \frac{\sum_i^I \frac{N_i}{N} \cdot \sigma_{s,g\to h,i}}{\sigma_{s,g}} 
           = \sum_i^I \frac{N_i}{N} \cdot P_{g\to h,i}
\end{equation}
The desired parameter, either $\sigma_{s,g\to h,i}$ or $P_{g\to h,i}$, lies within
the summation in equation \ref{P_serp}.

Therefore to obtain nuclide-specific scattering kernels Serpent must be prevented
from executing the above summation for all but the desired species.  Hence, the
appropriate loop was found and a material filter was added.  As a Monte Carlo code,
this filter would only be triggered if a scattering reaction occurred.  Thus the
addition of this filter did not adversely affect other tallies.  Moreover, this
filter was included after all physics calculations but prior to reporting.  The
underlying transport calculations are not altered by the use of the filter.

The implementation of the filter works as follows. First,the user specifies
a dummy filter material.  The nuclides present in this material are the species
which are tallied by the filter.  All other nuclides are ignored, as are the mass
fraction values.  Then the user sets the material to filter by via the new `gtpmat'
option in the Serpent code.  This has the syntax `set gtpmat $<$material label$>$'.

Serpent normally outputs the group transfer probability and the group-to-group
scattering cross section to the GTRANSFP and GTRANSFXS variables respectively.
However by using `gtpmat', GTRANSFP takes on the filtered value while GTRANSFXS
is invalidated.  The loss of meaning for GTRANSFXS here is not problematic
because, using the detector capability of Serpent, $\sigma_{s,g,i}$ may be computed.
This can then be convolved with the newly computed $P_{g\to h,i}$ to obtain
the correct $\sigma_{s,g\to h,i}$.

Since the filtered nuclides are given as a set (albeit as members of a material),
isotope-specific group transfer probabilities are obtained using materials with
only a single species.  This is the most common usage of the additional `gtpmat'
command.

\subsection{Group-to-Group Interpolations}

The primary purpose of storing sets of multigroup cross sections is
to be able to interpolate between them.  Since interpolation of existing
data is much faster than computing new values via neutron
transport methods, this allows a multigroup model to generate
run-time, reactor-state-specific group constants.  Such a model has been
demonstrated \cite{ScopatzAnthony2011}; in this reference, $\sigma_{s,g,i}$ was explicitly
verified, but the validity of interpolating the group-to-group scattering
cross section was only checked via its aggregation.  Here, interpolations on
$\sigma_{s,g\to h,i}$ are tested explicitly.

The state of the reactor and corresponding experimental variables are taken
to be the same as in \cite{ScopatzAnthony2011}.  Neutron energies were discretized 
into a 19-group structure designed to more finely capture the epithermal region.  
Call $E$ [MeV] a neutron energy. Then $E_g$ [MeV] denotes the energy group boundaries.
These are displayed in Table \ref{group_boundaries}.
\begin{table}[htbp]
\begin{center}
\caption{Energy Group Boundaries $E_g$}
\label{group_boundaries}
\begin{tabular}{|c|c||c|c|}
\hline
\textbf{$g$} & \textbf{$E_g$ [MeV]} & \textbf{$g$} & \textbf{$E_g$ [MeV]} \\
\hline
1  & 10      & 11 & 2.15E-5 \\
2  & 1       & 12 & 1.29E-5 \\
3  & 0.1     & 13 & 7.74E-6 \\
4  & 0.01    & 14 & 4.64E-6 \\
5  & 3.16E-3 & 15 & 2.78E-6 \\
6  & 1.00E-3 & 16 & 1.67E-6 \\
7  & 3.16E-4 & 17 & 1.00E-6 \\
8  & 1.00E-4 & 18 & 1.00E-7 \\
9  & 5.99E-5 & 19 & 1.00E-8 \\
10 & 3.59E-5 & 20 & 1.00E-9 \\
\hline
\end{tabular}
\end{center}
\end{table}

The unit cell considered here was a standard light-water reactor (LWR) taken from an
OECD/NEA Burnup Credit Criticality report \cite{Takano1994}.  The reactor was modeled as
an infinite lattice of these cells.  The benchmark (unperturbed) state of the
reactor may be seen in Table \ref{benchmark_rx_state}.
\begin{table}[htbp]
\begin{center}
\caption{Benchmark Reactor State}
\label{benchmark_rx_state}
\begin{tabular}{|l|c|c|}
\hline
\textbf{Parameter}            & \textbf{Symbol}      & \textbf{Value} \\
\hline
Fuel Density                  & $\rho_{\mbox{fuel}}$ & 10.7 g/cm\superscript{3}  \\
Cladding Density              & $\rho_{\mbox{clad}}$ & 5.87 g/cm\superscript{3}  \\
Coolant Density               & $\rho_{\mbox{cool}}$ & 0.73 g/cm\superscript{3}  \\
Fuel Cell Radius              & $r_{\mbox{fuel}}$    & 0.41 cm \\
Void Cell Radius              & $r_{\mbox{void}}$    & 0.4595 cm \\
Cladding Cell Radius          & $r_{\mbox{clad}}$    & 0.52165 cm \\
Unit Cell Pitch               & $\ell$               & 1.3127 cm \\
Number of Burn Regions        & $b_r$                & 10 \\
Fuel Specific Power           & $p_s$                & 0.04 MW/kgIHM \\
Initial \nuc{U}{235} Mass Fraction & $T_{\mbox{\nuc{U}{235}}0}$ & 0.04 kg\subscript{i}/kgIHM \\
Initial \nuc{U}{238} Mass Fraction & $T_{\mbox{\nuc{U}{238}}0}$ & 0.96 kg\subscript{i}/kgIHM \\
\hline
\end{tabular}
\end{center}
\end{table}

To test the validity of interpolating on the group-to-group scattering cross section,
the benchmark reactor was perturbed twice.  The fuel radius $r_{\mbox{fuel}}$ [cm]
was altered to ensure that non-trivial changes would be made to $\sigma_{s,g\to h,i}$
for at least some nuclides.  Perturbing only the fuel radius has the effect of
altering the fuel-to-moderator ratio, an important parameter in thermal reactors
such as LWRs.

Hence, the modified Serpent code was run for fuel radii of $\pm 10\%$
the benchmark value. The additional `gtpmat' card was used to create
group-to-group scattering cross sections for every nuclide in the burnup calculation.
Denote the $+10\%$ data with superscript ``$+$'' and the $-10\%$ data with the
superscript ``$-$''.  Also set the superscript ``$*$'' as data interpolated
from the previous two calculated sets.  A linear interpolation of the
scattering cross section is given simply as in equation \ref{lin_interp}:
\begin{equation}
\label{lin_interp}
\sigma_{s,g\to h,i}^{*} = \left(\sigma_{s,g\to h,i}^{+} - \sigma_{s,g\to h,i}^{-}\right)\cdot
                          \frac{r_{\mbox{fuel}}^{*} - r_{\mbox{fuel}}^{-}}
                               {r_{\mbox{fuel}}^{+} - r_{\mbox{fuel}}^{-}} +
                          \sigma_{s,g\to h,i}^{-}
\end{equation}
Such interpolations were carried out for three fuel radii: $-5\%$, $+5\%$, and the
benchmark fuel radius.

The interpolated scattering cross sections were then compared to the `true'
values (no superscript), as computed via Serpent. A fractional difference
$\epsilon$ [unitless] is defined as seen in equation \ref{epsilon_def}.
\begin{equation}
\label{epsilon_def}
\epsilon = \frac{\sigma_{s,g\to h,i}^{*}}{\sigma_{s,g\to h,i}} - 1
\end{equation}
Positive values of $\epsilon$ indicate that the interpolation over-predicts
the value of the scattering cross section for that $g$ \& $h$ combination.
Negative $\epsilon$ means that an under-prediction has occurred.  For this
interpolation technique to be valid, $\epsilon$ should be small for all
incident and exiting neutron energy groups.

While $\epsilon$ is sufficient for testing that $\sigma_{s,g\to h,i}$ is
valid for a specific $g$ \& $h$, it does not grant any insight into how the
interpolated scattering kernel performs as a whole.  For this reason, a summary
metric is desired.  Treating both $\sigma_{s,g\to h,i}^{*}$ and $\sigma_{s,g\to h,i}$
as independent measurements of the same underlying physical parameter, the degree of
likeness between them may be determined using Kendall's $\tau$.

Kendall's $\tau$ is a standard, unitless, statistical rank correlation coefficient.
It is defined on the range $[-1,1]$.  When $\tau=+1$ the inputs are in perfect agreement,
for $\tau=-1$ the measurements are in perfect disagreement, and with $\tau=0$ the
two inputs are fully independent of one another.  For the scattering kernel,
values closer to unity are desired.

The value $\tau$ is contructed by first sorting the values of both measurements.
For every pair of elements in both measurements, consistent strict monotonicity
is checked.  Thus $\tau$ is the number of stictly monotonic pairs minuse those
pairs which are not strictly monotonic divided by the total number or pairs.  
(Pairs which are equivalent are not counted.)  Here, the scattering kernels
were flattened prior to the th eevlauation of $\tau$.  Kendall's $\tau$ was 
computed using the ``\texttt{scipy.stats.kendalltau()}'' \cite{SciPy}.

For an explanation of why the Kendall $\tau$ test is necessary, consider the following
scenario.  Suppose a nuclide is found with an unacceptably large fractional
difference $\epsilon$ for some $g$ \& $h$.  If the value of $\tau$ is sufficiently
close to 1, this implies that the majority of the scattering cross section
remains valid.  Such a situation may arise for a heavy species when a perturbation 
to the configuration gives rise to a large change in the intragroup flux close to 
the group boundary.  Group boundaries may be refined to mitigate this.

Lastly, the interpolations and analysis were performed for the LWR only
at the beginning of life.  This was to minimize extraneous sources of error
coming from a multi-dimensional interpolation that includes fluence as well.  
In a burnup analysis, the interpolation on cross sections is also carried out 
on fluence and initial isotopic concentrations to capture transmutation feedbacks 
on the group constants -- see \cite{Scopatz2009} for details.

\section{RESULTS}
\label{sec:results}

Group-to-group scattering cross section interpolations were performed for
each nuclide for three fuel radii (0.3895, 0.41, and 0.4305 [cm]) in between two
fuel radii (0.369 and 0.451 [cm]) where data used in the interpolations was generated from the modified version of Serpent.  The interpolated values at the intermediate radii were then compared to the equivalent data
generated using Serpent.  Summary statistics are presented in
Tables \ref{r25}-\ref{r75}.
*** be sure to emphasize in table headings that these are the LARGEST error xfer functions per species ***
\begin{table}[htbp]
\begin{center}
\caption{Interpolation at $r_{\mbox{fuel}}=0.3895$ [cm]}
\label{r25}
\begin{tabular}{|l|ccccccc|}
\hline
nuclide & $\epsilon$ & $\tau$ & $g$ & $h$ & $\sigma_{s,g\to h,i}$ & $P_{g\to h,i}$ & $R_{a/s,g,i}$\\
\hline
CM250 & -0.308 & 0.849 & 18 & 18 & 26.879 & 0.667 & 4.304\\
H3 & 0.158 & 0.965 & 18 & 18 & 0.360 & 0.103 & 0.000\\
HE4 & 0.130 & 0.947 & 12 & 12 & 0.144 & 0.189 & 0.000\\
NA23 & -0.111 & 0.948 & 15 & 15 & 2.770 & 0.868 & 0.023\\
CM249 & 0.099 & 0.849 & 16 & 15 & 0.418 & 0.045 & 0.391\\
PU242 & 0.099 & 1.000 & 8 & 8 & 113.125 & 1.000 & 0.401\\
SM148 & -0.098 & 1.000 & 7 & 7 & 341.025 & 1.000 & 0.115\\
TH230 & -0.092 & 1.000 & 9 & 9 & 62.960 & 1.000 & 2.256\\
CM248 & 0.085 & 0.840 & 6 & 6 & 14.065 & 0.855 & 0.147\\
PM147 & 0.071 & 1.000 & 7 & 7 & 47.554 & 1.000 & 1.425\\
PU240 & 0.069 & 1.000 & 15 & 15 & 1151.980 & 1.000 & 11.956\\
ZR93 & 0.061 & 1.000 & 6 & 6 & 34.611 & 1.000 & 0.342\\
BI209 & -0.058 & 1.000 & 5 & 5 & 22.428 & 1.000 & 0.005\\
U236 & 0.055 & 1.000 & 8 & 8 & 44.080 & 1.000 & 0.696\\
PD107 & 0.054 & 1.000 & 6 & 6 & 8.789 & 1.000 & 2.036\\
CM244 & -0.052 & 1.000 & 7 & 7 & 27.099 & 1.000 & 0.949\\
PU246 & -0.051 & 0.863 & 0 & 0 & 2.514 & 0.735 & 0.600\\
BK249 & 0.050 & 0.877 & 14 & 14 & 15.150 & 0.882 & 10.669\\
BA133 & 0.043 & 1.000 & 8 & 8 & 4.891 & 1.000 & 7.083\\
BA140 & -0.041 & 1.000 & 6 & 6 & 300.010 & 1.000 & 0.039\\
U234 & -0.041 & 1.000 & 7 & 7 & 35.798 & 1.000 & 0.718\\
SM151 & -0.040 & 1.000 & 10 & 10 & 62.066 & 1.000 & 8.487\\
U232 & -0.039 & 1.000 & 11 & 11 & 12.300 & 1.000 & 23.553\\
TH232 & 0.035 & 1.000 & 7 & 7 & 59.285 & 1.000 & 0.440\\
TC99 & -0.035 & 0.999 & 12 & 12 & 24.519 & 1.000 & 26.507\\
RA226 & -0.033 & 1.000 & 5 & 5 & 36.568 & 1.000 & 0.121\\
PU244 & 0.032 & 1.000 & 8 & 8 & 22.236 & 1.000 & 1.482\\
SN125 & -0.032 & 1.000 & 6 & 6 & 19.538 & 1.000 & 0.301\\
I129 & -0.032 & 1.000 & 5 & 5 & 9.671 & 1.000 & 0.324\\
CS135 & 0.031 & 1.000 & 8 & 8 & 26.506 & 1.000 & 2.444\\
CM242 & 0.030 & 0.999 & 8 & 8 & 14.367 & 1.000 & 1.624\\
U239 & -0.026 & 1.000 & 11 & 11 & 25.202 & 1.000 & 6.969\\
PU239 & 0.026 & 1.000 & 7 & 7 & 17.769 & 1.000 & 4.075\\
AM243 & -0.024 & 1.000 & 15 & 15 & 67.047 & 1.000 & 35.991\\
PU238 & 0.021 & 1.000 & 7 & 7 & 17.274 & 1.000 & 1.137\\
SB126 & 0.019 & 1.000 & 12 & 12 & 5.847 & 1.000 & 30.736\\
U238 & -0.016 & 0.981 & 18 & 18 & 6.899 & 0.729 & 0.573\\
AM241 & -0.016 & 1.000 & 10 & 10 & 13.906 & 1.000 & 6.941\\
PU243 & -0.016 & 1.000 & 15 & 15 & 14.279 & 1.000 & 35.053\\
CM246 & 0.014 & 1.000 & 6 & 6 & 20.198 & 1.000 & 0.478\\
U235 & 0.014 & 0.986 & 17 & 18 & 4.380 & 0.283 & 29.265\\
H1 & 0.014 & 0.973 & 16 & 14 & 8.516 & 0.412 & 0.005\\
O16 & -0.013 & 0.978 & 18 & 18 & 1.897 & 0.383 & 0.000\\
CM243 & 0.011 & 1.000 & 14 & 14 & 8.926 & 1.000 & 108.925\\
\hline
\end{tabular}
\end{center}
\end{table}

\begin{table}[htbp]
\begin{center}
\caption{Interpolation at $r_{\mbox{fuel}}=0.41$ [cm]}
\label{r50}
\begin{tabular}{|l|ccccccc|}
\hline
nuclide & $\epsilon$ & $\tau$ & $g$ & $h$ & $\sigma_{s,g\to h,i}$ & $P_{g\to h,i}$ & $R_{a/s,g,i}$\\
\hline
H3 & -0.370 & 0.940 & 17 & 18 & 3.540 & 1.000 & 0.000\\
NA23 & -0.342 & 0.907 & 17 & 18 & 3.998 & 1.000 & 0.107\\
CM247 & -0.333 & 0.771 & 18 & 18 & 8.136 & 1.000 & 42.139\\
HE4 & -0.298 & 0.915 & 17 & 18 & 1.432 & 1.000 & 0.000\\
CF250 & 0.201 & 0.819 & 14 & 14 & 11.189 & 0.800 & 3.089\\
BK249 & 0.183 & 0.816 & 11 & 11 & 12.432 & 0.818 & 7.047\\
CM249 & 0.169 & 0.828 & 12 & 12 & 7.131 & 0.800 & 0.068\\
U234 & -0.126 & 1.000 & 7 & 7 & 39.440 & 1.000 & 0.688\\
U235 & -0.101 & 0.933 & 18 & 18 & 12.373 & 0.800 & 95.301\\
CF249 & -0.091 & 0.836 & 16 & 17 & 1.999 & 0.154 & 92.423\\
PM147 & 0.074 & 1.000 & 7 & 7 & 46.945 & 1.000 & 1.399\\
RA226 & 0.072 & 1.000 & 7 & 7 & 21.807 & 1.000 & 0.504\\
TH230 & -0.071 & 1.000 & 9 & 9 & 61.967 & 1.000 & 2.253\\
BI209 & -0.071 & 1.000 & 5 & 5 & 23.266 & 1.000 & 0.005\\
PU246 & -0.064 & 0.762 & 17 & 17 & 10.823 & 1.000 & 50.433\\
PD107 & 0.060 & 1.000 & 8 & 8 & 40.963 & 1.000 & 2.575\\
SM148 & -0.057 & 1.000 & 5 & 5 & 47.611 & 1.000 & 0.058\\
CM244 & 0.056 & 1.000 & 7 & 7 & 24.577 & 1.000 & 0.891\\
PU242 & -0.043 & 1.000 & 8 & 8 & 129.052 & 1.000 & 0.403\\
U238 & 0.042 & 0.990 & 17 & 18 & 2.231 & 0.236 & 0.199\\
SM151 & -0.038 & 1.000 & 10 & 10 & 62.391 & 1.000 & 8.446\\
PU239 & -0.036 & 1.000 & 8 & 8 & 17.050 & 1.000 & 5.778\\
BA140 & -0.034 & 1.000 & 6 & 6 & 301.337 & 1.000 & 0.039\\
TC99 & 0.031 & 0.999 & 6 & 6 & 8.639 & 1.000 & 1.161\\
U232 & -0.030 & 1.000 & 11 & 11 & 12.279 & 1.000 & 23.644\\
CM242 & 0.028 & 0.999 & 8 & 8 & 14.097 & 1.000 & 1.593\\
PU238 & 0.028 & 0.999 & 7 & 7 & 17.260 & 1.000 & 1.143\\
U236 & 0.028 & 1.000 & 5 & 5 & 24.191 & 1.000 & 0.193\\
SB126 & 0.028 & 0.999 & 12 & 12 & 5.736 & 1.000 & 30.370\\
U239 & -0.028 & 0.999 & 10 & 10 & 19.133 & 1.000 & 4.392\\
PU244 & 0.027 & 0.999 & 7 & 7 & 19.555 & 1.000 & 1.016\\
PU240 & 0.026 & 1.000 & 5 & 5 & 17.817 & 1.000 & 0.337\\
SN125 & -0.025 & 1.000 & 5 & 5 & 7.239 & 1.000 & 0.174\\
TH232 & -0.025 & 1.000 & 6 & 6 & 26.832 & 1.000 & 0.519\\
PA231 & 0.024 & 1.000 & 8 & 8 & 16.375 & 1.000 & 2.894\\
U237 & 0.023 & 0.999 & 8 & 8 & 19.622 & 1.000 & 2.453\\
CM246 & 0.022 & 1.000 & 7 & 7 & 25.524 & 1.000 & 1.083\\
CS136 & -0.022 & 1.000 & 9 & 9 & 40.377 & 1.000 & 0.367\\
H1 & -0.022 & 0.943 & 17 & 18 & 52.634 & 0.740 & 0.008\\
EU155 & 0.022 & 1.000 & 9 & 9 & 21.953 & 1.000 & 3.211\\
ZR93 & 0.021 & 1.000 & 3 & 3 & 9.271 & 1.000 & 0.047\\
I129 & 0.020 & 1.000 & 7 & 7 & 8.220 & 1.000 & 1.366\\
TH228 & -0.020 & 1.000 & 14 & 14 & 76.426 & 1.000 & 22.781\\
AM241 & 0.018 & 1.000 & 15 & 15 & 15.096 & 1.000 & 49.952\\
\hline
\end{tabular}
\end{center}
\end{table}

\begin{table}[htbp]
\begin{center}
\caption{Interpolation at $r_{\mbox{fuel}}=0.4305$ [cm]}
\label{r75}
\begin{tabular}{|l|ccccccc|}
\hline
nuclide & $\epsilon^{\max}$ & $\tau$ & $g^{\max}$ & $h^{\max}$ & $\sigma_{s,g\to h,i}^{\max}$ & $P_{g\to h,i}^{\max}$ & $R_{a/s,g,i}^{\max}$\\
\hline
HE4 & -0.115 & 0.963 & 16 & 16 & 0.409 & 0.537 & 0.000\\
H3 & -0.113 & 0.975 & 19 & 19 & 1.184 & 0.336 & 0.000\\
U235 & -0.094 & 0.976 & 19 & 19 & 12.22 & 0.791 & 94.32\\
PU242 & 0.090 & 1.000 & 7 & 7 & 16.25 & 1.000 & 0.463\\
U236 & -0.080 & 0.999 & 8 & 8 & 47.23 & 1.000 & 0.691\\
PU240 & -0.054 & 1.000 & 9 & 9 & 76.04 & 1.000 & 1.482\\
U238 & 0.046 & 0.989 & 19 & 19 & 6.467 & 0.684 & 0.573\\
CM242 & 0.042 & 0.999 & 9 & 9 & 13.61 & 1.000 & 1.487\\
PU238 & 0.038 & 1.000 & 8 & 8 & 17.20 & 1.000 & 1.144\\
PD107 & 0.035 & 1.000 & 9 & 9 & 42.11 & 1.000 & 2.572\\
ZR93 & -0.034 & 1.000 & 7 & 7 & 38.54 & 1.000 & 0.347\\
EU155 & 0.033 & 1.000 & 11 & 11 & 13.26 & 1.000 & 9.100\\
SM151 & -0.032 & 1.000 & 13 & 13 & 22.55 & 1.000 & 13.40\\
I129 & 0.031 & 1.000 & 8 & 8 & 8.145 & 1.000 & 1.326\\
NP237 & 0.028 & 1.000 & 9 & 9 & 16.35 & 1.000 & 3.932\\
TC99 & 0.027 & 1.000 & 13 & 13 & 23.88 & 1.000 & 26.40\\
CM244 & 0.025 & 1.000 & 8 & 8 & 25.57 & 1.000 & 0.919\\
U234 & -0.023 & 1.000 & 13 & 13 & 183.4 & 1.000 & 7.085\\
PU239 & 0.022 & 1.000 & 8 & 8 & 17.90 & 1.000 & 4.138\\
AM243 & -0.021 & 1.000 & 16 & 16 & 66.88 & 1.000 & 35.69\\
CM246 & -0.019 & 1.000 & 6 & 6 & 23.25 & 1.000 & 0.263\\
H1 & -0.019 & 0.971 & 17 & 19 & 11.60 & 0.163 & 0.005\\
O16 & -0.017 & 0.979 & 19 & 19 & 1.874 & 0.378 & 0.000\\
CS135 & -0.014 & 1.000 & 7 & 7 & 21.05 & 1.000 & 0.363\\
CS137 & -0.012 & 1.000 & 5 & 5 & 12.07 & 1.000 & 0.006\\
AM242 & -0.010 & 1.000 & 13 & 13 & 11.06 & 1.000 & 18.19\\
KR85 & -0.007 & 1.000 & 5 & 5 & 7.971 & 1.000 & 0.079\\
AM241 & -0.006 & 1.000 & 11 & 11 & 13.90 & 1.000 & 6.961\\
PU241 & 0.005 & 1.000 & 11 & 11 & 26.38 & 1.000 & 11.38\\
CM245 & -0.004 & 1.000 & 9 & 9 & 11.27 & 1.000 & 7.979\\
CM243 & -0.002 & 1.000 & 12 & 12 & 12.78 & 1.000 & 18.90\\
CS134 & -0.002 & 1.000 & 1 & 1 & 3.983 & 1.000 & 0.006\\
SR90 & -0.002 & 1.000 & 4 & 4 & 7.052 & 1.000 & 0.003\\
SE79 & 0.001 & 1.000 & 1 & 1 & 2.840 & 1.000 & 0.005\\
\hline
\end{tabular}
\end{center}
\end{table}
 *** move at least some of this above the tables as it helps explain what readers will see***
 *** the discussion below may change once you toss out really low-concentration species***
 *** can you report how many xfer functions have epsilon < say 0.02?  e.g. this would be nice:  "5,500 of 5,600 nonzero scattering kernel entries had errors of less than 2%."***
These summary tables contain all of the nuclides Serpent may natively compute
(\emph{i.e.} those present in ENDF/B-VII \cite{}).  The nuclides are sorted from
highest-to-lowest maximum absolute value of the fractional difference $\epsilon$
for all $g$ \& $h$.  Thus the top entry gives a maximal error for all group
constants for all nuclides.  Nuclides whose $\mbox{max}(|\epsilon|)$ is less than
that of the last entry are not displayed.  The value of this $\epsilon$ is
shown in the second column.

For the vast majority of species in these three tables the interpolated value of
the group-to-group scattering cross section is within $\pm 5\%$ of the measured
value.  Thus in most cases $\epsilon$ is enough to conclude that interpolation
is a valid run-time technique for generating scattering kernels.

Furthermore many species have fractional differences within the 2\% mark.
It is especially important to note that \nuc{H}{1} and \nuc{O}{16} lie in
this regime even at thermal energies where changes to the fuel-to-moderator ratio have a large effect on coarse-group cross sections.  Hydrogen in an LWR is the primary mechanism of neutron
moderation.  

However, some nuclides have fractional differences which are above the
\emph{a priori} acceptability mark.  For such species, Kendall's $\tau$ is
shown in the third column of Tables \ref{r25}-\ref{r75}.  The value of
$\tau$ is always greater than $0.75$ and is overwhelmingly much closer to 1.
This result implies that even for nuclides which have a $\sigma_{s,g\to h,i}$ which
is significantly off, that the off indices $g$ \& $h$ are relatively isolated.
The rest of the scattering kernel for such species remains valid.

That said, the group constants whose interpolations do lead to significant error
demand investigation.  Columns 4 and 5 display the incident and exiting indices
$g$ \& $h$ at which the maximum absolute $\epsilon$ occurs.  Column 6 shows
the Serpent-computed value of $\sigma_{s,g\to h,i}$ for these indices while column 7
shows the corresponding group transfer probability.  Lastly, column 8 presents
$R_{a/s,g,i}$ [unitless], or the ratio of the absorption cross section to the
scattering cross section for the $g$\superscript{th} group.  Namely:
\begin{equation}
\label{R_as}
R_{a/s,g,i} = \frac{\sigma_{a,g,i}}{\sigma_{s,g,i}}
\end{equation}
*** again, this may change as I bet you those with big errors will fall off the table.  You can just say that larger errors arise in certain very low-abundance species ***
Some species, such as \nuc{Cm}{247} in Table \ref{r75}, have large fractional deviations
but also large $R_{a/s,g,i}$.  In these cases, the indicated errors in the
scattering cross section are not important because the reaction rate is dominated
by absorption rather than scattering.  For \nuc{Cm}{247} here, an incident neutron is
over 42 times more likely to be captured than to scatter into any other energy.
Therefore, the errors in the scattering kernel are insignificant to the performance
of the reactor as a whole.  This argument holds for all species with $1 << R_{a/s,g,i}$.

Still, some nuclides have relatively high $\epsilon$ values and
low absorption-to-scattering ratios.  Prime examples are \nuc{H}{3} and \nuc{He}{4}
in Table \ref{r25}.  This is because these species have effectively no
absorption cross section.  However, here the group transfer probabilities
for these $g$ \& $h$ combinations are low.  This indicates that an incident
neutron will likely not scatter into the $h$\superscript{th} group.  Therefore the
relatively higher error in this part of the group-to-group scattering cross section
is less important to the overall reactor as well.

*** actually, Na-23 is too light to be a fission product.  You can probably just delete this paragraph ***
Thus the problematic species which remain are those which have $P_{g\to h,i}$ close
to 1, $R_{a/s,g,i} < 1$, and high $\epsilon$.  A representative of this class is
\nuc{Na}{23} in Table \ref{r50}.  However, such nuclides are generally only present
in an LWR is trace amounts.  Their low number density pushes down their reaction rate
and thus their importance.  For example, \nuc{Na}{23} generally only exists in an
LWR as a fission product.  (Sodium, being highly chemically reactive with water, is
generally left out as a design material.)

%\clearpage

\section{CONCLUSIONS \& FUTURE WORK}
\label{sec:conclusions}

This study demonstrates the use of the Monte Carlo neutron transport code Serpent to generate group-to-group scattering
cross sections.  This capability enables the construction of a multi-energy
group cross section libraries.  Such libraries may then be used by a reactor
model to form state-specific cross sections at run time.

In a prior study, the reactor state specific cross sections were found
via linear interpolations.  Previously, only $\sigma_{s,g,i}$ was
explicitly verified.  In this paper, the fuel pin radius was varied.
Interpolations of $\sigma_{s,g\to h,i}$ were taken and compared against
the results of additional Serpent runs for matching $r_{\mbox{fuel}}$.

For most species, interpolated group-to-group scattering cross sections
were found to be well within acceptable error.  *** I deleted a bunch of text here that reviewed what was said right above.  Instead, can you add a couple of sentences describing convergence studies to establish how coarsely-spaced the explicit calculations must be to ensure some level of fidelity?  That's probably the key bit of future work someone will have to do...***

Thus this paper shows that the group-to-group scattering cross sections may be
both generated using Serpent for specific isotopes as well as interpolated for use in a fast multigroup solver.  Given low-error burnup-dependent cross section sets, a multigroup solver can achieve results of fidelity comparable to coupled transport-burnup calculations.

*** deleted last paragraph because transport will still be needed to get the known flux spectrum.  So, it wouldn't help much! ***


\section*{ACKNOWLEDGEMENTS}
The authors would like to thanks Jaakko Lepp\"{a}nen for his guidance
on Serpent's architecture.


%
\setlength{\baselineskip}{12pt}
\bibliographystyle{physor}
\bibliography{library}

\end{document}




